\documentclass[a4paper,10pt]{article} %can change the doc wide font size here if needed

%-------------%
%Packages
%-------------%

%document formatting packages
\usepackage{fancyhdr}
\usepackage{ragged2e}
\usepackage{float}
\usepackage[makeroom]{cancel}
\usepackage[margin=1in]{geometry} %can be useful to change if you have large tables or figures
\usepackage{lipsum}
\usepackage{pdfpages}
\usepackage{titlesec}
\usepackage{pdflscape}


%Table packages
\usepackage{colortbl}
\usepackage{xcolor}
\usepackage{booktabs}
\usepackage{longtable}
\usepackage{threeparttable}
\usepackage{multirow}
\usepackage{tabularx}
\usepackage{multirow}
\usepackage{array}
\usepackage{makecell}

%Math rendering related packages
\usepackage{amsmath}
\usepackage{physics}
\usepackage{gensymb}
\usepackage{amssymb}
\usepackage{siunitx}
\usepackage{textcomp}
\AtBeginDocument{\RenewCommandCopy\qty\SI}

%figure generation/management packages
\usepackage{chemfig}
\usepackage{wrapfig}
\usepackage{placeins}
\usepackage{graphicx}

%pgf plots setup
\usepackage{pgfplots}
\pgfplotsset{compat=1.18, width=10cm}


%list creation packages
\usepackage{enumerate}
\usepackage{enumitem}

%referencing setup/style
\usepackage[style=apa, backend=biber]{biblatex}
\addbibresource{Refs.bib}

%for making the links you can press on for references and tables etc.
\usepackage{hyperref}
\hypersetup{
    colorlinks=true,
    linkcolor=blue,
    filecolor=magenta,      
    urlcolor=cyan,
    pdftitle={Overleaf Example},
    pdfpagemode=FullScreen,
    }

\titleformat{\paragraph}[block]
  {\normalfont\normalsize\bfseries}  % formatting of text
  {\theparagraph}                    % numbering
  {1em}                              % spacing between number and title
  {}                                 % before-code (empty)
\titlespacing*{\paragraph}
  {0pt}{3.25ex plus 1ex minus .2ex}{1ex plus .2ex}


%---------------------------%
% Preamble
%---------------------------%





\begin{document}

\pagestyle{fancy}
\fancyhf{}
\rfoot{Page \thepage}
\renewcommand{\headrulewidth}{0pt}
\fancypagestyle{plain}{%
  \fancyhf{}
  \rfoot{Page \thepage}
  \renewcommand{\headrulewidth}{0pt}
}

\newenvironment{FRpage}{%
  \clearpage
  \begin{landscape}
  \thispagestyle{plain}
  \begingroup
  \footnotesize
  \setlength{\tabcolsep}{3.5pt} % narrower column padding
}{%
  \endgroup
  \end{landscape}
  \clearpage
}

%---------------------------%
% Coverpage
%---------------------------%

\author{Kristian Stracke, Bailey Firman, Angus Liaubon, William Collins, Adam Blunt}\\
\date{\today\\[2ex]
Bachelor of Engineering (Mechanical) – BH070 \\
\bigskip
Systems Engineering Principles MIET2385 – Professor John Mo}


\title{\includegraphics*[scale=0.1]{rmit-university-logo.png} \\[14ex] \textbf{Strap Launcher Detailed Design}} %This renders the graphics for the logo




\setcounter{secnumdepth}{4}  % allow numbering down to \paragraph
\setcounter{tocdepth}{4}     % include paragraphs in table of contents

\maketitle
\thispagestyle{empty}
\pagenumbering{roman}
\setcounter{page}{0}

\newpage
\cleardoublepage
\tableofcontents
\cleardoublepage
\listoftables
\cleardoublepage
\listoffigures

\newpage

\justifying

\pagenumbering{arabic}
\setcounter{page}{1}

%---------------------------%
% Document
%---------------------------%


\section{Introduction}

\subsection{Addressing Previous Deliverables}

The first deliverable saw a self-formed group of students collaborating on developing an effective solution for assisting with strapping loads on trucks and trailers. With information provided with the scenario depictions and the supplied Q and A, the team was able to develop a solution that is viable for the local market and use with the farmer's patented technology, the strap weight.  \bigskip

\noindent The proposed solution saw a device utilising a flywheel system to propel the strap and weight over the load put onto the trucks and trailers. The deliverable saw the conversion of user requirements to system requirements, assessment of requirements and development of a solution to fix the problem within all relevant legal parameters, even developing some elements of detailed design of a functioning prototype. \bigskip 

\noindent The second deliverable in relation to this prototype saw the same team who developed the theoretical functioning prototype had to pitch the product in a presentation. This pitch saw typical elements of a sales pitch in terms of the product expectations and capabilities, with less focus on the financials of the solution. \bigskip   

\subsection{Consolidation of Client Requirements}

After having investigated all related provided information from the farmer and delivery drivers consulted in regards to their requirements, their needs have been categorised below:

\begin{table}[h]
\caption{Consolidation of Client Requirements}
\centering
\renewcommand{\arraystretch}{1.4} % Adds a bit of breathing room
\setlength{\tabcolsep}{6pt}       % Adjust cell padding if needed
\begin{tabular}{
    |>{\centering\arraybackslash}m{0.5\textwidth}|
     >{\centering\arraybackslash}m{0.5\textwidth}|
}
\hline
\textbf{Client Requirements} & \textbf{System Capabilities} \\ \hline
Launch strap over loads 5.0m tall and 2.4m wide & System designed with sufficient propulsion to clear max legal load of 5.5m high and 2.7m wide, with elements of resistance taken into account for calculations such as wind speed. \\ \hline
Less than 10kg system weight & A lightweight handheld solution has been prioritised for the solution. \\ \hline
Portable system preferred & Design of a lightweight handheld system has been prioritised for the solution. \\ \hline
Easy to use & Developing a simple user interface and easy operational procedure for simplicity. \\ \hline
Secure individual strap within 2 minutes  & Having easy operational procedure with quick reset time between launches. \\ \hline
Small enough to store in truck cabin ("no bigger than a sports bag") & Compact system design, utilising generic physical construction to allow for fitting in more convenient locations where required. \\ \hline
No "excessive labour" to operate & Incorporated in the lightweight handheld design ideals, and the easy operation ideals. \\ \hline
Able to be used by sole operator & Having the system only needing the handheld component of the device to propel the strap, without need of a target projection system on the other side. \\ \hline
Usable in varying weather conditions (wind, rain and heat) & Material selection to be suitable to typical weather conditions seen in Australian farming environments throughout the year. \\ \hline
Use patented strap weight in design of solution &
The housing area for the object to be propelled has been considered to be the strap weight. \\ \hline
\end{tabular}
\label{tab:ClientRequirements}
\end{table}

%Insert table cross referencing all related points from documents depicting needs with how they have been met.

\newpage

\section{Summary of Functions and Sub-Systems}

\subsection{Functions}

\subsubsection{Prepare Device}

The purpose of this function is to take the device from its stowed configuration into its ready to load configuration.

\paragraph{Process of function}


\begin{itemize}
    \item Inspect device and make sure it is visibly in working order with no obvious damage.
    \item Connect the appropriate power/pneumatic connection, depending on the users needs in the situation.
    \item Onboard controller tests for any electrical faults or sensor lock outs that would prevent the device from operating properly and more importantly preventing putting the operator and/or device in danger.
    \item Onboard controller signals an LED to display the readiness state of the device clearly to the operator.
\end{itemize}

\paragraph{Performance Requirements for function}

\begin{itemize}
    \item Setup completed in $<60$ Seconds by single operator.
    \item Diagnostics check $<5$ Seconds. 
\end{itemize}


\subsubsection{Launch Strap}

The purpose of this function is to prepare the device to fire the strap over the load. 

\paragraph{Process of function}


\begin{itemize}
    \item Onboard controller validates the interlocks and pressure/charge condition of the device.
    \item Trigger initiates the launching sequence for the strap.
    \item Either the flywheel or the pneumatic piston accelerates the strap through the launch chamber.
    \item Damping assembly mitigates the recoil to $< 50N$
\end{itemize}

\paragraph{Performance Requirements for function}

\begin{itemize}
    \item Launch Height for a load of $\geq$ 5 Meters high and, 2.7 Metres in width.
    \item Emergency cutoff halts operation if a fault is detected.
\end{itemize}

\newpage

\subsubsection{Maintenance/Fault Reporting}

The purpose of this function is to log information about the devices operation to the memory on the onboard controller that would be relevant to maintaining the device. i.e; The number of launches since last service, The total number of launches and cycles of the battery. As well as logging any faults the device may have encountered.

\paragraph{Process of function}

\begin{itemize}
    \item If a fault is detected the onboard controller will determine if it is critical and prevent operation or if it is minor and allow operation possibly in a limp mode or simply log the error for maintenance,
    \item Record the launch count in that specific cycle and add it to the total number.
    \item Record the battery cycle count.
\end{itemize}

\paragraph{Performance Requirements for function}

\begin{itemize}
    \item Operate with 100\% reliability with exact launch and cycle counts.
    \item Prevent the device from operating if there is an outstanding maintenance issue or an exceptional circumstance presents itself.
\end{itemize}





\subsubsection{Stow Device}

The purpose of this function is to return the device to its safe state ready to store for the next time it is needed.

\paragraph{Process of function}


\begin{itemize}
    \item Disarm the power system, Bleed any leftover air in system.
    \item Onboard controller logs the number of launches or any faults that need to be reported for maintainence.
    \item Remove chargeable battery to be placed in cradle.
    \item Fold down device so it is returned to it's starting position.
\end{itemize}

\paragraph{Performance Requirements for function}

\begin{itemize}
    \item Stowed in $<90$ Seconds by single operator.
    \item Display charging state of battery with LED indicator.
    \item No residual pressure in pneumatic system.
\end{itemize}


\newpage

\subsection{Subsystem Explanation}

\subsubsection{Propulsion and Power}

\paragraph{Primary Power} 

The Strap Propulsion Device (SPD) primarily uses an electric motor flywheel system to function. This system works by the motor speeding up a flywheel to store rotational energy, which is then used to propel the strap clasp from its resting position. This method ensures dependable performance across various loads and operational scenarios without the need for external assistance. From the user's standpoint, the handheld launcher design simplifies aiming and controlling the strap's path, reducing the physical exertion typically associated with manual throwing. The launch speed can be modified via motor input control, allowing users to adjust propulsion based on different load sizes and weather conditions. A digital display offers real-time data on motor speed and power settings, enabling users to verify launch parameters before activation. The device is powered by either a rechargeable onboard battery pack or a standard 12/24 V vehicle connection. A docking station provides easy recharging between uses, minimising downtime and eliminating the hassle of frequent battery replacements.


\paragraph{Alternative Power} 

Though primarily intended for electric propulsion, the SPD is equipped with a straightforward pneumatic backup for scenarios where electricity access is restricted, such as in secluded farms or extended field applications. This backup system incorporates a compressed air canister alike those found in nail guns or portable compressors, which is directly linked to a small piston mechanism within the launcher. Upon activation, the compressed air is released into the piston chamber, propelling the launch mechanism forward and launching the strap. A pressure regulator ensures the launch force stays within safe operating thresholds. While this pneumatic option serves as an emergency fallback, it lacks the precision and adjustability of the electric system.


\subsubsection{User Interface and Control}

The User Interface establishes the boundary for operators when setting up, arming, and launching systems, featuring controls that can be operated with gloves, such as an arm/disarm switch, a protected momentary trigger, and a toggle for handheld or mounted mode. A display that is readable in daylight conditions shows critical information like motor RPM, the energy stored in the flywheel, the status of the battery or a 12/24 V vehicle power supply, and the projected clearance or overshoot based on the angles from sensors and the current velocity setting. Launch velocity is selected via preset levels with options for precise adjustments. Upon arming, the subsystem performs a methodical pre-launch checklist: verifying battery charge or external power, ensuring the motor operates within speed and thermal constraints, checking the engagement of the clutch and holding latch, confirming the muzzle angle is within a safe range, and affirming the exclusion zone status. Only when all conditions are met does the trigger become active, with any inhibiting factors communicated through straightforward diagnostic messages. If the pneumatic backup mode is chosen, the display adjusts to show line pressure and valve condition, prevents firing below a specified minimum pressure, and notes the predicted trajectory based on pressure and angle data. Charging and docking functions are incorporated, showing charge rates and estimated readiness times to minimise downtime. Automatic timeouts revert the subsystem to a safe state after extended inactivity. The overall design of the User Interface aims to reduce cognitive load through a straightforward set $\rightarrow$ arm $\rightarrow$ fire process, ensuring the operator receives ample feedback to meet clearance and overshoot requirements with minimal physical effort.

\subsubsection{Safety}

The Safety subsystem is responsible for isolating energy and managing potential hazards in both propulsion modes. It necessitates a series of hardware checks before launch can proceed: ensuring safe elevation and azimuth, completing the arming process, verifying latch and clutch integrity, engaging mount locks, and confirming that flywheel speed and temperature are within acceptable ranges. Additionally, for pneumatic backup, it checks for minimum line pressure and a reliable valve path. The safety relay only closes when all these checks are satisfied, and any failure results in a latched inhibit with a specific trip code. An Emergency Stop works by de-energising to trip; it halts motor drive, applies the flywheel brake, disengages the clutch and holding latch, and opens a normally-open vent to release air—loss of power triggers the same safety responses (default-to-safe). Recoil is capped at 50 N or less; if exceeded, re-arming is blocked until rectified. The subsystem reports permissive and trip status to the User Interface for display and logging but does not rely on this for stopping the plant. This configuration meets the safety requirements for interlock, emergency stop, recoil control, and exclusion-zone protection within the specified environmental conditions.

\subsubsection{Payload and Strap Loading}

The SPD is engineered to accommodate standard ratchet straps commonly used in trucking and farming, regardless of whether strap weights are attached. These straps are placed into a propulsion clasp lock that firmly holds the buckle until the launch is activated. This clasp is compatible with typical buckle types to ensure maximum compatibility. The loading procedure is designed to be straightforward and consistent, enabling an operator to quickly prepare multiple straps.

\subsubsection{Launch Mechanism} %Needs paraphrasing

The launch system mainly comprises a user-controlled trigger and a safety switch. In electric mode, the launch process starts when the safety switch is turned off and the trigger is pressed, causing the flywheel to accelerate to the selected speed. Pressing the trigger again engages the flywheel, releasing the strap. Alternatively, in pneumatic backup mode, once the canister is loaded and the safety switch is turned off, the strap is immediately ready to fire when the trigger is pressed.

To manage recoil in electric mode, power selection settings are designed to ensure that no more than 50N of force is exerted on the user. In pneumatic mode, a bleed valve releases any pressure that exceeds the safety threshold, thus preventing excessive recoil. Additionally, for user comfort while using the SPD, a rubber butt stock is attached to absorb any felt recoil.

\subsubsection{System Fault Diagnostics} %Needs paraphrasing

The SPD is designed to be a dependable tool, with fault diagnostics capabilities allowing operators to swiftly pinpoint and fix issues on-site. This feature aids in system maintenance and is closely tied to data logging. Sensors are strategically placed in key components where faults are most likely to arise, such as the payload area, battery and power connections, and the motor controller. These sensors detect strap jams, electrical issues, or motor failures, and in pneumatic backup mode, a pressure sensor ensures proper pressurisation before launch. When a fault is identified, the data is transmitted to the user display with a clear popup message, consistent with fault reporting, facilitating rapid troubleshooting without the need for specialised training.

\newpage

\subsection{Test Plans}



\newpage

\section{Numerical Proof of Functions}

\subsection{Launch energy required to satisfy requirements}

\subsubsection{Geometry and variables of the situation}

\begin{itemize}
    \item Clearance requirement $>$5 Metres in Height ($H$), $>$2.7 Metres in width ($R$)
    \item Angle of launch ($\theta$)
    \item Mass of payload; Buckle + Specific weight of strap ($m$)
    \item Environmental Variables; Wind, Rain
\end{itemize}

\subsubsection{Ideal situation}

Here we will calculate a perfect world situation where we can ignore any variables such as wind resistance operator error etc, using the method derived from \parencite{MunganCarlE.2017OtLo} accessed via the RMIT Library.

\begin{equation}
y(x)=x\tan\theta-\frac{g x^{2}}{2 v_0^{2}\cos^{2}\theta}.
\end{equation}

\ldots taking \(y(R)=H\) gives

\begin{equation*}
H = R\tan\theta-\frac{g R^{2}}{2 v_0^{2}\cos^{2}\theta}
\Rightarrow\quad
v_0^{2} = \frac{g R^{2}}{2}\,\frac{1}{\cos^{2}\theta\,(R\tan\theta-H)}
\end{equation*}

\ldots let \(u=\tan\theta\) so that \(\sec^{2}\theta=1/\cos^{2}\theta=1+u^{2}\)

\begin{equation*}
v_0^{2}=\frac{g R^{2}}{2}\;\frac{1+u^{2}}{R u - H}
\end{equation*}

\ldots for a given (R,H), the minimum $v_0$ occurs at.

\begin{equation*}
    f(u) = \frac{1 + u^2}{Ru-H}, \qquad u>\frac{H}{R}
\end{equation*}

\ldots minimise \(f(u)=\dfrac{1+u^{2}}{R u - H}\). Differentiate and set to zero:

\medskip

\begin{equation*}
    f'(u)=\frac{(2u)(Ru-H)-(1+u^{2})R}{(Ru-H)^{2}}=0
    \;\;\Rightarrow\;\; Ru^{2}-2Hu-R=0 \\
\end{equation*}

\ldots solve the quadratic

\begin{equation*}
u=\tan\theta
= \frac{H+\sqrt{H^{2}+R^{2}}}{R}.
\end{equation*}

\ldots substitute \(u\) back into \(v_0^{2}\) and simplify to obtain the minimum launch velocity:

\begin{equation}
\boxed{\,v_{0,\min}=\sqrt{\,g\big(\sqrt{R^{2}+H^{2}}+H\big)}\,}
\end{equation}

\ldots then the required angle:

\begin{equation}
\boxed{\,\tan\theta=\dfrac{\sqrt{R^{2}+H^{2}}+H}{R}}
\end{equation}

Now that we have the equations we can input our values to find that: $v_{0,min} \approx 10.258 m\cdot s^{-1}$, $\theta \approx 70.67 ^\circ$ additionally we can also quickly calculate the flight time simply by dividing the vertical component by the velocity and the cosine of the launch angle:

\begin{equation}
    t_{flight} = \frac{R}{v_0 \cos\theta} \approx 2.07s
\end{equation}

\newpage

\begin{figure}[htbp!]
\caption{Trajectory diagram comparing minimum velocity to design velocity}
\centering
\begin{tikzpicture}
  \begin{axis}[
    width=17cm, height=11cm,
    axis lines=middle,
    xlabel={$x$ (m)}, ylabel={$y$ (m)},
    xmin=-3, xmax=9.0,      % show the -2 m standoff
    ymin=0.0,  ymax=9,      % set baseline so the green launch sits at y=0
    domain=0:10.0,
    samples=400,
    grid=both,
    every axis plot/.append style={thick}
  ]
    % Parameters
    \pgfmathsetmacro{\g}{9.81}    % gravity
    \pgfmathsetmacro{\yzero}{1.0} % launch height set to ground (y=0)
    \pgfmathsetmacro{\D}{2.0}     % standoff to near edge (m) from x=0
    \pgfmathsetmacro{\W}{2.7}     % truck width (m)
    \pgfmathsetmacro{\Htop}{5.0}  % truck height (m)

    % Launch parameters
    \pgfmathsetmacro{\theta}{70.672} % deg
    \pgfmathsetmacro{\vzero}{10.258} % m/s (blue)
    \pgfmathsetmacro{\vone}{12.4}    % m/s (green)
    \pgfmathsetmacro{\thetac}{cos(\theta)}

    % GREEN TRAJECTORY OFFSET: launch from x = -2 m
    \pgfmathsetmacro{\xoffset}{-2.0}

    % Blue trajectory (origin at x=0, y=0)
    % y(x) = y0 + x tanθ - g x^2 /(2 v0^2 cos^2θ)
    \addplot[blue,domain=0:10]
      ({x},
       {\yzero + x*tan(\theta) - (\g*x^2)/(2*\vzero*\vzero*\thetac*\thetac)})
      node[pos=0.65, above right] {\small Blue: launch at $x{=}0$ m};

    % Green trajectory starting at x = -2 m, baseline y0=0
    % y(x) = y0 + (x - x0) tanθ - g (x - x0)^2 /(2 v1^2 cos^2θ), for x >= x0
    \addplot[green!70!black,domain=\xoffset:10]
      ({x},
       {\yzero + (x-\xoffset)*tan(\theta)
         - (\g*(x-\xoffset)^2)/(2*\vone*\vone*\thetac*\thetac)})
      node[pos=0.62, above right];

    % Launch points
    \addplot[only marks, mark=*, mark size=2pt, blue]  coordinates {(0,\yzero)};
    \addplot[only marks, mark=*, mark size=2pt, green!70!black] coordinates {(\xoffset,\yzero)};
    \node[anchor=north west, blue] at (axis cs:0.1,\yzero) {\small Minimum $(0,0)$};
    \node[anchor=north west, green!60!black] at (axis cs:\xoffset+0.1,\yzero)
      {\small Design $(-2,0)$};

    % Truck body rectangle [D, D+W] x [0, Htop]
    \addplot[fill=gray!25, draw=gray]
      coordinates {
        (\D,0)
        (\D+\W,0)
        (\D+\W,\Htop)
        (\D,\Htop)
      } -- cycle;

    % Edges and labels
    \draw[dashed, gray] (axis cs:\D,0) -- (axis cs:\D,\Htop);
    \draw[dashed, gray] (axis cs:\D+\W,0) -- (axis cs:\D+\W,\Htop);
    \node[anchor=west, gray!60!black] at (axis cs:\D+\W,2.5) {\small Truck body};

    % Clearance check points at near and far edges
    \addplot[only marks, mark=*, mark size=1.8pt, red]
      coordinates {(\D,\Htop) (\D+\W,\Htop)};
    \node[anchor=south, red] at (axis cs:\D,\Htop+0.4) {\tiny Near edge};
    \node[anchor=south, red] at (axis cs:\D+\W,\Htop+0.4) {\tiny Far edge};
  \end{axis}
\end{tikzpicture}
\label{fig:trajectory}
\end{figure}

\noindent Figure \ref{fig:trajectory} compares the design trajectory velocity to the minimum required velocity. As shown the design velocity provides a much larger margin for error to account for variables such as oversized loads, adverse weather conditions such as wind as well as letting the operator stand farther back from the vehicle. 

\medskip

The green line shows a velocity of $12.4m/s$ with a standoff distance of 4 metres from the body of the truck. The blue line shows a velocity of $10.258m/s$ with a standoff distance of 2 metres from the truck. 

\subsubsection{Assumptions regarding the trajectory}

In order to calculate the trajectory we need to make some assumptions, In this example the standoff distance is 2 and 4 metres from the truck body. This is also assuming that the operator is of the average male height and is holding the device roughly 1 metre in the air.

\medskip

Since this is the minimum velocity required a safe estimate would be to add 20\% onto this value to consistently clear the load. This provides a design velocity of \textbf{$\approx 12.4m/s$}

\newpage

\subsection{Launch energy}

\subsubsection{Launch energy requirements}

\begin{equation}
    E = \frac{1}{2}mv^2_0 \quad [J]
\end{equation}

\ldots kinetic energy of the launch

\begin{equation}
    J = mv_0 \quad [N \cdot s]
\label{eqn:impulse}
\end{equation}

\ldots specific impulse of the launch

\begin{equation}
    v_{recoil} = \frac{J}{m_{device}+m_{strap}} \quad [m/s]
\end{equation}

\ldots velocity of the recoil produced

\begin{equation}
    F_{peak} \approx \frac{J}{\Delta t} \quad [N]
    \label{eqn:fpeak}
\end{equation}

\ldots peak force generated by the launch is dependent on the time over which the force damper acts (iterated on in Table \ref{tab:recoil_forces})

\subsubsection{Iteration on the masses of the projectiles}


\begin{table}[ht]
\centering
\caption{Projectile energy, impulse and device recoil}
\label{tab:proj_energy_recoil}
\begin{tabular}{@{}rrrrr@{}}
\toprule
Projectile mass $m$ [kg] & $E=\tfrac12 m v_0^2$ [J] & $J=m v_0$ [N$\cdot$ s] & {$v_{\rm rec}=J/m_{\rm}$ [m/s]} & $E_{\rm rec}=\tfrac12 m_{\rm} v_{\rm rec}^2$ [J] \\
\midrule
0.20 & 15.38 & 2.480 & 0.827 & 1.025 \\
0.40 & 30.75 & 4.960 & 1.653 & 4.100 \\
0.60 & 46.13 & 7.440 & 2.480 & 9.226 \\
\bottomrule
\end{tabular}
\begin{tablenotes}[flushleft]
\footnotesize
\textit{Note}. Device mass $(m_{\rm}) = 9\,\mathrm{kg}$.
\end{tablenotes}
\end{table}

\begin{table}[ht]
\centering
\caption{Recoil force iterated [N]}
\label{tab:recoil_forces}
\begin{threeparttable}
\begin{tabular}{@{}rccc@{}}
\toprule
Projectile mass $m$ [kg] & $\Delta t=0.05$\,$s$ & $\Delta t=0.10$\,$s$ & $\Delta t=0.20$\,$s$ \\
\midrule
0.20 & 49.6 & 24.8 & 12.4 \\
0.40 & 99.2 & 49.6 & 24.8 \\
0.60 & 148.8 & 74.4 & 37.2 \\
\bottomrule
\end{tabular}
\begin{tablenotes}[flushleft]
\footnotesize
\textit{Note}. Derived from Equation \ref{eqn:fpeak} and Equation \ref{eqn:impulse}.
\end{tablenotes}
\end{threeparttable}
\end{table}

\newpage


\subsection{Recoil mitigation}

Recoil can be mitigated with a few techniques. The use of a damper, which is generally an arrangement of springs mounted in the stock. The other being increasing the weight of the device, For this instance both methods are used. Firstly a pilot mass is projected over the load. This is patented by the farmer and has a weight of $\approx 480g$. As per Table \ref{tab:recoil_forces} that recoil is within range.

\bigskip

Focusing on this value for the mass of the projectile.


\begin{align*}
    J &= mv_0\\
    J &= 0.48 \cdot 12.4\\
    J &= 5.952 N \cdot s
\end{align*}

\ldots we fined the impulse for this weight to be $J = 5.952 N\cdot s$. Now using Equation \ref{eqn:fpeak} Iterate the impulse over time.

\begin{table}[ht]
\centering
\caption{Actual recoil force [N]}
\label{tab:recoil_forces}
\begin{threeparttable}
\begin{tabular}{@{}rccc@{}}
\toprule
Projectile mass $m$ [kg] & $\Delta t=0.05$\,$s$ & $\Delta t=0.10$\,$s$ & $\Delta t=0.20$\,$s$ \\
\midrule
0.48 & 49.6 & 24.8 & 12.4 \\
\bottomrule
\end{tabular}
\begin{tablenotes}[flushleft]
\footnotesize
\textit{Note}. Derived from Equation \ref{eqn:fpeak} and Equation \ref{eqn:impulse}.
\end{tablenotes}
\end{threeparttable}
\end{table}

For the minimum damping time\ldots

\begin{equation}
    F \approx \frac{J}{\Delta t} \Rightarrow \Delta t \geq \frac{J}{F_{Max}}
\end{equation}

\begin{equation*}
    \Delta t \geq \frac{5.952}{50} = 0.119s
\end{equation*}


\newpage

\subsection{Flywheel Design}

For the electrical propulsion it is difficult for a small electric motor to impart the required amount of force on the projectile over such a short distance of action. Similarly as to how a bow or crossbow works, storing mechanical potential energy in a string allows the wielder to impart far more energy as opposed to simply throwing the arrow. Here we will apply the same principle. In this instance a flywheel will be utilised.

\subsubsection{Flywheel Design Considerations}

There are a couple factors to consider when designing the flywheel:

\begin{itemize}
    \item The energy required to be stored to meet the criteria
    \item The size of the flywheel
    \item The material of the flywheel
    \item The bearing and the support structure; How does it interface with the rest of the device
    \item Stress concentrations around the bore and fillets
\end{itemize}

\subsubsection{Flywheel Energy Calculations}

Firstly the projectile energy needs to be calculated\ldots

\begin{equation}
    E_p = \frac{1}{2}m_pv^2_0 \qquad [J]
\end{equation}

\begin{equation*}
    E_p = 36.9J
\end{equation*}

\ldots the flywheel needs to at minimum store 36.9 Joules, although after accounting for aero, grip, slip and frictional losses, assume that only 70\% of the energy will be transferred:

\begin{equation}
    \Delta E_{FW} = \frac{E_p}{\eta}
\end{equation}

\begin{equation*}
    \Delta E_{FW} = 52.7J, \qquad \eta = 0.70
\end{equation*}

\noindent Taking this value the inertia, speeds and dimensions of the flywheel can be found, A speed drop of 75\% will be assumed to keep the flywheel light as well as lower speeds for safety\ldots

\begin{equation}
    \Delta E_{FW} = \frac{1}{2}(\omega^2_1-\omega^2_2) \Rightarrow \frac{1}{2}I(1-0.25)\omega^2_1, \qquad     \omega_2 = 0.25\omega_1
\end{equation}

\begin{equation*}
    \Delta E_{FW} = 0.46875 I \omega^2_1
\end{equation*}

\ldots with the target $\Delta E_{FW} = 52.6J$ use $\omega_1 = 8000 RPM = 837.76 rad \cdot s^{-1}$

\begin{equation*}
    I = \frac{52.7}{0.46875 I \omega^2_1} = 1.60 \times 10^{-4} kg \cdot m^2
\end{equation*}

\ldots find the disc mass from inertia, choose an example thickness of $5mm$ this leaves room for machining:

\begin{equation}
    I = \frac{1}{2}\rho \pi r^4 t \Rightarrow r = \sqrt[\leftroot{-3}\uproot{3}4]{\frac{2 I}{\rho \pi t}}
\end{equation}

\begin{equation*}
    r = 40.21mm
\end{equation*}

\newpage

\subsubsection{Flywheel Calculations Summary}

For this flywheel a solid design will be used to save on machining cost space as well as increasing the structural rigidity of the flywheel. 304 Stainless will be used as the material for its cheap cost, high strength and corrosion resistance to meet the users environmental requirements

\begin{itemize}
    \item Radius: $r = 40.21mm$
    \item $\rho_{304} = 8000kg/m^3$
    \item Mass: $m = \rho \pi r^2 t = 0.201$
\end{itemize}

\paragraph{Material stresses}

For the flywheels hoop stresses at $8000RPM$ the rim speed is\ldots

\begin{equation}
    v = r\omega = 33.48m \cdot s^{-1}
\end{equation}

\ldots therefore the thin rim hoop estimated stress is:

\begin{equation}
    \sigma \approx \rho v^2 \approx 9.0\text{MPa}
\end{equation}

\noindent 304 Stainless has a reported Yield strength ($S_y$) $\approx 205\text{MPa}$ \textcite{DMConsultancy304} the stresses in the flywheel are a fraction of this value.

\newpage

\subsection{Clutch Design}


\newpage

\subsection{Motor Design}

\newpage

\subsection{Damper Design}

\newpage

\section{Assembly Drawings}

\newpage

\section{Systems Engineering Management Plan (SEMP)}

\subsection{Scope and Objectives}



\subsection{Roles and Responsibilities}



\newpage

\subsection{Work Breakdown Structure}

%Table here to categorise all work done

\begin{table}[h!]
\centering
\caption{Work Breakdown} \medskip
\begin{tabular}{|c|l|l|l|l|c|}
\hline
Name & \multicolumn{1}{c|}{\begin{tabular}[c]{@{}c@{}}Bailey\\ Firman\end{tabular}} & \multicolumn{1}{c|}{\begin{tabular}[c]{@{}c@{}}Kristian\\ Stracke\end{tabular}} & \multicolumn{1}{c|}{\begin{tabular}[c]{@{}c@{}}William\\ Collins\end{tabular}} & \multicolumn{1}{c|}{\begin{tabular}[c]{@{}c@{}}Angus\\ Liaubon\end{tabular}} & \multicolumn{1}{c|}{\begin{tabular}[c]{@{}c@{}}Adam\\ Blunt\end{tabular}} \\ \hline
Contribution & s4036816 & s4037579 & s4037366 & s4036652 & s3906627 \\ \hline
1 &  &  &  &  &  \\ \hline
2 &  &  &  &  &  \\ \hline
3 &  &  &  &  &  \\ \hline
4 &  &  &  &  &  \\ \hline
5 &  &  &  &  &  \\ \hline
6 &  &  &  &  &  \\ \hline
7 &  &  &  &  &  \\ \hline
8 &  &  &  &  &  \\ \hline
9 &  &  &  &  &  \\ \hline
10 &  &  &  &  &  \\ \hline
\end{tabular}
\label{tab:WorkBreakdown}
\end{table}

\underline{\textbf{Tasks:}}
\smallskip
\begin{enumerate}[label=\arabic*.]
    \item Introduction
    
    \begin {enumerate}[label*=\arabic*]
    
        \item Addressing Previous Deliverables
        
        \item Consolidation of Client Requirements
        
    \end{enumerate}

    \item Summary of Functions and Sub-Systems

    \begin{enumerate}[label*=\arabic*]
    
        \item Functions

        \item Subsystem Explanation

        \item Test Plans
        
    \end{enumerate}

    \item Numerical Proof of Functions
    
    \begin{enumerate}[label*=\arabic*.]
    
        \item Launch Energy Required to Satisfy Requirements

        \item Launch Energy

        \item Recoil Mitigation

        \item Flywheel Design

        \item Clutch Design

        \item Motor Design

        \item Damper Design

    \end{enumerate}

    \item Assembly Drawings

    \item Systems Engineering Management Plan (SEMP)

    \begin{enumerate}[label*=\arabic*]
    
        \item Scope and Objectives

        \item Roles and Responsibilities

        \item Work Breakdown Structure

        \item Risk Management

        \item Verification and Validation Plan

        \item Configuration Management

        \item Quality Management

        \item Safety Management

        \item Maintenance and Support Plan
        
    \end{enumerate}

    \item Appendices

    \begin{enumerate}[label*=\arabic*]
    
        \item FFBDs

        \item Function Tree

        \item UR-SR Traceability Matrix

        \item Interface Definitions

        \item Test Procedures
        
    \end{enumerate}
    
\end{enumerate}

\newpage

\subsection{Schedule and Milestones}

\subsection{Risk Management}

\subsection{Verification and Validation Plan}

\subsection{Configuration Management}

\subsection{Quality Management}

\subsection{Safety Management}

\subsection{Maintenance and Support Plan}

\newpage

\printbibliography

\newpage

\appendix

\section{FFBDs}

\section{Function Tree}

\section{UR--SR Traceability Matrix}

\section{Interface Definitions}

\section{Test Procedures}

\end{document}

